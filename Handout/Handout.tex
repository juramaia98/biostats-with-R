\documentclass[]{tufte-handout}

% ams
\usepackage{amssymb,amsmath}

\usepackage{ifxetex,ifluatex}
\usepackage{fixltx2e} % provides \textsubscript
\ifnum 0\ifxetex 1\fi\ifluatex 1\fi=0 % if pdftex
  \usepackage[T1]{fontenc}
  \usepackage[utf8]{inputenc}
\else % if luatex or xelatex
  \makeatletter
  \@ifpackageloaded{fontspec}{}{\usepackage{fontspec}}
  \makeatother
  \defaultfontfeatures{Ligatures=TeX,Scale=MatchLowercase}
  \makeatletter
  \@ifpackageloaded{soul}{
     \renewcommand\allcapsspacing[1]{{\addfontfeature{LetterSpace=15}#1}}
     \renewcommand\smallcapsspacing[1]{{\addfontfeature{LetterSpace=10}#1}}
   }{}
  \makeatother

\fi

% graphix
\usepackage{graphicx}
\setkeys{Gin}{width=\linewidth,totalheight=\textheight,keepaspectratio}

% booktabs
\usepackage{booktabs}

% url
\usepackage{url}

% hyperref
\usepackage{hyperref}

% units.
\usepackage{units}


\setcounter{secnumdepth}{-1}

% citations

% pandoc syntax highlighting
\usepackage{color}
\usepackage{fancyvrb}
\newcommand{\VerbBar}{|}
\newcommand{\VERB}{\Verb[commandchars=\\\{\}]}
\DefineVerbatimEnvironment{Highlighting}{Verbatim}{commandchars=\\\{\}}
% Add ',fontsize=\small' for more characters per line
\newenvironment{Shaded}{}{}
\newcommand{\AlertTok}[1]{\textcolor[rgb]{1.00,0.00,0.00}{\textbf{#1}}}
\newcommand{\AnnotationTok}[1]{\textcolor[rgb]{0.38,0.63,0.69}{\textbf{\textit{#1}}}}
\newcommand{\AttributeTok}[1]{\textcolor[rgb]{0.49,0.56,0.16}{#1}}
\newcommand{\BaseNTok}[1]{\textcolor[rgb]{0.25,0.63,0.44}{#1}}
\newcommand{\BuiltInTok}[1]{#1}
\newcommand{\CharTok}[1]{\textcolor[rgb]{0.25,0.44,0.63}{#1}}
\newcommand{\CommentTok}[1]{\textcolor[rgb]{0.38,0.63,0.69}{\textit{#1}}}
\newcommand{\CommentVarTok}[1]{\textcolor[rgb]{0.38,0.63,0.69}{\textbf{\textit{#1}}}}
\newcommand{\ConstantTok}[1]{\textcolor[rgb]{0.53,0.00,0.00}{#1}}
\newcommand{\ControlFlowTok}[1]{\textcolor[rgb]{0.00,0.44,0.13}{\textbf{#1}}}
\newcommand{\DataTypeTok}[1]{\textcolor[rgb]{0.56,0.13,0.00}{#1}}
\newcommand{\DecValTok}[1]{\textcolor[rgb]{0.25,0.63,0.44}{#1}}
\newcommand{\DocumentationTok}[1]{\textcolor[rgb]{0.73,0.13,0.13}{\textit{#1}}}
\newcommand{\ErrorTok}[1]{\textcolor[rgb]{1.00,0.00,0.00}{\textbf{#1}}}
\newcommand{\ExtensionTok}[1]{#1}
\newcommand{\FloatTok}[1]{\textcolor[rgb]{0.25,0.63,0.44}{#1}}
\newcommand{\FunctionTok}[1]{\textcolor[rgb]{0.02,0.16,0.49}{#1}}
\newcommand{\ImportTok}[1]{#1}
\newcommand{\InformationTok}[1]{\textcolor[rgb]{0.38,0.63,0.69}{\textbf{\textit{#1}}}}
\newcommand{\KeywordTok}[1]{\textcolor[rgb]{0.00,0.44,0.13}{\textbf{#1}}}
\newcommand{\NormalTok}[1]{#1}
\newcommand{\OperatorTok}[1]{\textcolor[rgb]{0.40,0.40,0.40}{#1}}
\newcommand{\OtherTok}[1]{\textcolor[rgb]{0.00,0.44,0.13}{#1}}
\newcommand{\PreprocessorTok}[1]{\textcolor[rgb]{0.74,0.48,0.00}{#1}}
\newcommand{\RegionMarkerTok}[1]{#1}
\newcommand{\SpecialCharTok}[1]{\textcolor[rgb]{0.25,0.44,0.63}{#1}}
\newcommand{\SpecialStringTok}[1]{\textcolor[rgb]{0.73,0.40,0.53}{#1}}
\newcommand{\StringTok}[1]{\textcolor[rgb]{0.25,0.44,0.63}{#1}}
\newcommand{\VariableTok}[1]{\textcolor[rgb]{0.10,0.09,0.49}{#1}}
\newcommand{\VerbatimStringTok}[1]{\textcolor[rgb]{0.25,0.44,0.63}{#1}}
\newcommand{\WarningTok}[1]{\textcolor[rgb]{0.38,0.63,0.69}{\textbf{\textit{#1}}}}

% longtable

% multiplecol
\usepackage{multicol}

% strikeout
\usepackage[normalem]{ulem}

% morefloats
\usepackage{morefloats}


% tightlist macro required by pandoc >= 1.14
\providecommand{\tightlist}{%
  \setlength{\itemsep}{0pt}\setlength{\parskip}{0pt}}

% title / author / date
\title{Biostats with R}
\author{Aditya Srinivasulu}
\date{10 January 2020}


\begin{document}

\maketitle




\hypertarget{what-is-r}{%
\section{What is R?}\label{what-is-r}}

Defined by the R Project for Statistical Computing as a language and
environment for statistical computing and graphics (R Core Team,
\protect\hyperlink{ref-R-base}{2016}), R is an open-source free language
primarily designed to be used to perform statistics.

It is a suite of tools and facilities for data manipulation, analysis,
and visualisation. The three main features of R are:

\begin{itemize}
\item
  It is an effective data handling and manipulation facility
\item
  It is a modular collection of functions and packages that allow
  processing and analysis
\item
  It is a simple, readable, yet effective and fully-functioning
  programming language
\end{itemize}

In recent times, R has seen a surge in popularity. Rexer, a company
which handle big data analysis and also conduct annual surveys of data
scientists worldwide, found in 2015 that R is the most common
primary-use statistical package, and this has remained the case since
(Rexer, \protect\hyperlink{ref-rexer2015}{2015}).

\begin{center}\rule{0.5\linewidth}{\linethickness}\end{center}

\hypertarget{starting-off-in-r}{%
\section{Starting off in R}\label{starting-off-in-r}}

To begin working in R, we have to first install R and RStudio, a helpful
Integrated Development Environment (IDE, essentially a software that
makes it easier to work with a programming language).

\hypertarget{installing-r}{%
\subsection{Installing R}\label{installing-r}}

First, head to the Comprehensive R Archive Network (CRAN)
website\footnote{\url{https://cloud.r-project.org/}} and click the link
to download the appropriate version of R for your operating system.

\textbf{On Windows:}

\begin{itemize}
\item
  \emph{Ensure you download the \textbf{base} version!}
\item
  When the file is downloaded, find it in the Downloads section of your
  browser, and click it to open. Alternatively, find it in the Downloads
  folder of your computer, and double-click it to open.
\item
  Follow the default instructions and complete the installation.
\end{itemize}

\textbf{On Mac:}

\begin{itemize}
\item
  Download the .pkg file
\item
  When the file is downloaded, find it in the Downloads section of your
  browser, and click it to open. Alternatively, find it in the Downloads
  folder of your computer, and double-click it to open.
\item
  Follow the default instructions and complete the installation.
\end{itemize}

\textbf{On Linux:}

Follow Jason French's guide\footnote{\url{https://www.jason-french.com/blog/2013/03/11/installing-r-in-linux/}}
on how to install R on Linux.

\hypertarget{installing-rstudio}{%
\subsection{Installing RStudio}\label{installing-rstudio}}

Once you've installed R, head to the RStudio download page\footnote{\url{https://rstudio.com/products/rstudio/download/}}
and scroll down till you find the list of ``Installers for Supported
Platforms'' for RStudio Desktop. Download the appropriate installer and
run it, keeping everything to default.

\begin{center}\rule{0.5\linewidth}{\linethickness}\end{center}

\hypertarget{speaking-in-r}{%
\section{`Speaking' in R}\label{speaking-in-r}}

\hypertarget{objects}{%
\subsection{Objects}\label{objects}}

R treats data in the form of objects, called \textbf{vectors} (we will
refer to them as objects, however). These objects can have values or
data assigned to them.

R can be used as a complex calculator:

\begin{Shaded}
\begin{Highlighting}[]
\DecValTok{1} \OperatorTok{+}\StringTok{ }\DecValTok{26} \OperatorTok{*}\StringTok{ }\NormalTok{(}\DecValTok{152}\OperatorTok{/}\DecValTok{3}\NormalTok{)}
\end{Highlighting}
\end{Shaded}

\begin{verbatim}
## [1] 1318.333
\end{verbatim}

And you can create new objects using the assignment symbol\footnote{The
  shortcut in RStudio for this is \texttt{Alt\ +\ -}},
\texttt{\textless{}-}

\begin{Shaded}
\begin{Highlighting}[]
\NormalTok{object <-}\StringTok{ }\DecValTok{2} \OperatorTok{+}\StringTok{ }\DecValTok{2}
\NormalTok{data <-}\StringTok{ "This object contains some data"}
\end{Highlighting}
\end{Shaded}

You can always remember that \texttt{\textless{}-} means \texttt{gets}.
So \texttt{object\ \textless{}-\ value} is read as
\texttt{object\ gets\ value}.

You can call objects that you have assigned data to by typing in their
name:

\begin{Shaded}
\begin{Highlighting}[]
\NormalTok{object}
\end{Highlighting}
\end{Shaded}

\begin{verbatim}
## [1] 4
\end{verbatim}

\begin{Shaded}
\begin{Highlighting}[]
\NormalTok{data}
\end{Highlighting}
\end{Shaded}

\begin{verbatim}
## [1] "This object contains some data"
\end{verbatim}

Object names can be anything as long as they start with a letter and
have no special characters in them except \texttt{.} and \texttt{\_}. I
personally use a mix of \texttt{snake\_case\_object\_names} and
\texttt{CamelCaseObjectNames} but it's good to be consistent.

Also, very importantly: \textbf{R is case-sensitive}. In the words of
Hadley Wickham, there is an implied contract between you and R - it will
do all the tedious computation for you, but in return you must be
completely precise in your instructions (Wickham and Grolemund,
\protect\hyperlink{ref-WickhamR4DS}{2016}).

\hypertarget{functions}{%
\subsection{Functions}\label{functions}}

R interprets, manipulates, and analyses data using \textbf{functions}.
Functions are typically in the format
\texttt{function(object,\ arguments)} where \texttt{function} is the
name of the function (e.g.~\texttt{mean()}, \texttt{sum()}),
\texttt{object} is the name of the object which holds the data in
question, and \texttt{arguments} are any arguments that the function
requires to\ldots{} function.

For instance, \texttt{rnorm} is a function that generates a random
normal distribution. It doesn't need an object, but it does need three
arguments: \texttt{n} for the sample size, \texttt{mean} for the mean of
the distribution, and \texttt{sd} for the standard deviation.

\begin{Shaded}
\begin{Highlighting}[]
\KeywordTok{rnorm}\NormalTok{(}\DataTypeTok{n =} \DecValTok{4}\NormalTok{, }\DataTypeTok{mean =} \FloatTok{3.4}\NormalTok{, }\DataTypeTok{sd =} \DecValTok{1}\NormalTok{)}
\end{Highlighting}
\end{Shaded}

\begin{verbatim}
## [1] 3.515278 3.833365 2.666358 2.345179
\end{verbatim}

However, in order to work with the results of a function downstream, we
must store it in an object. We use the assigner \texttt{\textless{}-}
for that. Remember that the assigner is read as \texttt{gets}.

\begin{Shaded}
\begin{Highlighting}[]
\NormalTok{normal <-}\StringTok{ }\KeywordTok{rnorm}\NormalTok{(}\DataTypeTok{n =} \DecValTok{300}\NormalTok{, }\DataTypeTok{mean =} \DecValTok{24}\NormalTok{, }\DataTypeTok{sd =} \DecValTok{6}\NormalTok{)}
\end{Highlighting}
\end{Shaded}

After storing data in an object, we can analyse it further - let's use
the \texttt{mean()} and \texttt{sd()} functions to find the mean and sd
of our random normal distribution. Then, let's use the
\texttt{boxplot()} function to make a simple boxplot of our
distribution.

\begin{Shaded}
\begin{Highlighting}[]
\KeywordTok{mean}\NormalTok{(normal)}
\end{Highlighting}
\end{Shaded}

\begin{verbatim}
## [1] 23.83508
\end{verbatim}

\begin{Shaded}
\begin{Highlighting}[]
\KeywordTok{sd}\NormalTok{(normal)}
\end{Highlighting}
\end{Shaded}

\begin{verbatim}
## [1] 6.103753
\end{verbatim}

\begin{Shaded}
\begin{Highlighting}[]
\KeywordTok{boxplot}\NormalTok{(normal)}
\end{Highlighting}
\end{Shaded}

\begin{marginfigure}
\includegraphics{Handout_files/figure-latex/unnamed-chunk-6-1} \caption[A simple box plot using the 'base' graphics in R]{A simple box plot using the 'base' graphics in R}\label{fig:unnamed-chunk-6}
\end{marginfigure}

\hypertarget{refs}{}
\leavevmode\hypertarget{ref-R-base}{}%
R Core Team (2016) \emph{R: A language and environment for statistical
computing}. Vienna, Austria: R Foundation for Statistical Computing.
Available at: \url{https://www.R-project.org/}.

\leavevmode\hypertarget{ref-rexer2015}{}%
Rexer (2015) `Data science survey'. Available at:
\url{https://www.rexeranalytics.com/data-science-survey.html}.

\leavevmode\hypertarget{ref-WickhamR4DS}{}%
Wickham, H. and Grolemund, G. (2016) \emph{R for data science}.
California: O'Reilly Media.



\end{document}
